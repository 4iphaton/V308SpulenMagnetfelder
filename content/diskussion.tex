\section{Diskussion}
\label{sec:Diskussion}

Bei der Messung von Magnetfelder kann es zu diversen Fehlern kommen, welche das Ergrbnis
der Messungen von der Theorie abweichen lässt.
Allgemein lässt sich sagen, dass Spulen einen Innenwiderstand haben, der den Strom eventuell
schwächt und somit auch die magnetische Flussdichte. Zudem heißen sich die Drähte
von Spulen mit erhöhter Nutzung auf, was diesen Innenwiderstand noch erhöht.
Außerdem wurden noch weitere Messungen mit Magnetfeldern durch eine andere Gruppe
durchgeführt, was ebenfalls die gemessene magnetische Flussdichte beeinflussen kann.
Zuletzt könnte die verwendete Hall-Sonde, egal ob longitudinal oder transversal, die
relative Permeabilität beeinflusst haben, wodurch sich die Werte ebenfalls verändern.

Sämtliche Messungen wurden mit halber maximaler Stromstärke und mit maximaler Stromstärke durchgeführt und die
sich ergebende magnetische Flussdichte ist in allen Messungen $ I \propto B$ entsprechend.
Ein Vergleich der beiden Messungen der langen Spule mit der Theorie zeigt, dass
Biot-Savart das äußere $B$ einigermaßen gut darstellt, obwohl dieses nur für Leiterschleifen
mit kaum Ausdehnung genutzt werden sollte.  Für die Formel mit der langen Spule zeigt sich, dass
der Wert zu niedrig ist, was sich durch die Permeablitität Hall-Sonde erklären lässt. Zudem ist zu beachten,
dass diese theoretischen Formel nur eine Näherung und keine exakte Beschreibung ist.
Wenn man gleiches für die kleine Spule tut stellt man fest, dass Biot-Savart für diese
in ihrem Mittelpunkt und innerhalb geringere Werte annimmt als die experimentell gemessenen, was erneut durch die Hall-Sonde
erklärt werden könnte. Im äußeren Bereich beschreibt Biot-Savart den Verlauf aber wieder vergleichsweise gut.
Um zu zeigen, dass die Formel der langen Spule für de kurze keine gute Näherung ist
wurde diese Formel ebenfalls verwendet, aber es zeigt sich, dass sie nicht mit den Messwerten übereinstimmt.

Für die Felder innerhalb der Helmholtz-Spulen zeigt sich, dass die Messwerte im Vergleich
zur Theoriekurve niedriger sind. Selbiges gilt für das äußere Feld, obwohl die Werte sich dort nicht
so stark unterscheiden. Dies könnte auf den Innenwiderstand der Helmholtzspulen zurückzuführen sein.

Bei der Hysteresekurve fällt die magnetische Flussdichte sehr schnell, sodass bereits nach einer Sekunde längerer
Messzeit viel kleinere oder größere Werte erhalten werden. Dies liegt and der Methode des Ablesens durch die Hallsonde.
Die geforderten Werte für Koerzitivkraft, Remanenz und Sättigungsmagnetisierung, zwar sehr klein, aber  realistisch.
